% Chapter Template

\chapter{Conclusions} % Main chapter title

\label{Conclusions} % Change X to a consecutive number; for referencing this chapter elsewhere, use \ref{ChapterX}

 %----------------------------------------------------
 
 
With the progress of the digital and manufacturing technologies available, it is useful to know how these can be integrated into the practice of dentistry, to identify the phases in which it is possible to integrate and to evaluate whether these have a practical benefit for the patient.\\
The possible applications of additive manufacturing in dentistry are many, given the large number of procedures in which the dentist is committed to designing and creating customized products for the patient. The practical advantages of digital technologies are primarily the reduction of the materials and the space necessary for taking the impression, which being digital can be easily stored indefinitely in an archive, and if necessary printed, even after years without no deterioration. \\
The same digital impression can easily be exchanged between the dentist and the technician, facilitating the exchange of information and reducing the possibility of errors on the part of the operator.\\
However, the possibilities for personalizing treatment are the most interesting feature of the digital approach to dentistry. The possibility of creating customized orthodontic brackets in the studio, which can be further modified to manage each treatment step, is certainly an interesting perspective that deserves further study. \\ The realization of anatomical post-extraction implants in the studio has already been tested in at least two trials. clinical \parencite{Reference85}, \parencite{Reference87} and, even if performed with different techniques, shows how the custom alternative to the classic fixture is a real possibility. This approach has shown potential to be a valuable resource in some rehabilitative situations, so further studies are needed to validate its effectiveness and provide operational guidelines for clinical use.\\
Prosthodontic dentistry has been using CAD-CAM technologies for some time during some phases of the treatment. From this perspective, additive manufacturing technologies must be evaluated in order to find real advantages towards subtractive manufacturing. The saving of material is certainly an advantage of additive manufacturing, as well as the ability to create extremely complex shapes. Precision is a fundamental element in prosthetics, and from the studies analyzed it is clear that in many circumstances the precision and accuracy achieved by additive manufacturing are at the level of subtractive or slightly more accurate manufacturing, in a range generally considered suitable for clinical use. \\
The possibility of performing an intraoral scan and comparing it digitally with the previous ones is interesting, and can be extremely useful in monitoring the growth of the little patient, especially in orthodontics. \\
With the dentist who is often already in the conditions to use these technologies it is important to promote the competence in the management of digital data. The guarantee of patient privacy is of primary importance, given the ease with which digital data can be exchanged. Proper management of medical images and models is important to preserve the original information as unaltered as possible. Finally, the printing process must be accurate, with great regard to the operating condition of the printer and to the printing parameters. The literature provides general knowledge and useful insights, but these must always be integrated with the manufacturer's instructions and with the specifications of the machine.
Future perspectives include the printing of scaffolds for tissue regeneration and, further, the in vitro production of tissues and organs ready for transplantation on the patient. Both of these strands of research use additive manufacturing techniques, and there are already those who suggest that in the future tissue printing can be integrated into the clinical routine \parencite{Reference142}. \\
These are certainly stimulating prospects and deserve a more careful discussion, due to the various implications these technologies may have in the future. The technology progresses quickly and with it the related machinery and software, so the methods discussed here represent only one of the many approaches that can be used by the clinician. The patient is always at the center of treatment, therefore every therapeutic choice must be carefully evaluated and put in place only if this brings about a real improvement in the patient's quality of life.
