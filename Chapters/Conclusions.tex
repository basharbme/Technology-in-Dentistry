% Chapter Template

\chapter{Conclusions} % Main chapter title

\label{Conclusions} % Change X to a consecutive number; for referencing this chapter elsewhere, use \ref{ChapterX}

 %----------------------------------------------------
 
 
With the progress in digital planning and automated manufacturing technologies, it is useful to know how these tools can be integrated into the practice of dentistry and to evaluate whether these have a practical benefit for the patient.\\
There are many applications of additive manufacturing in dentistry, given the large number of procedures in which the dentist is committed to designing and creating customized products for the patient. The practical advantages of digital technologies are primarily in the reduction of the materials and the space necessary to obtain the impression of the patient's dental arches. In fact digital impressions can be easily stored indefinitely in an archive and printed when it is necessary, even after years. \\
The same digital impression can easily be shared between the dentist and the technician, facilitating the exchange of information and reducing the possibility of errors.\\
However, the possibilities for customized treatment is the most interesting feature of the digital approach to dentistry. The possibility of creating customized orthodontic brackets, which can be further modified to manage each treatment step is certainly an interesting perspective that deserves further study. \\ The fabrication of root-analog post-extractive implants in studio has already been tested in at least two clinical trials \parencite{Reference85}, \parencite{Reference87}, and even if performed with different techniques, it shows how the custom alternative to the classic fixture is a real possibility. This approach has shown potential to be a valuable resource in some rehabilitative situations, so further studies are needed to validate its effectiveness and provide operational guidelines for clinical use.\\
Prosthodontic dentistry has already been using CAD-CAM technologies during some phases of the treatment. From this perspective, additive manufacturing technologies must be evaluated in order to find real advantages towards subtractive manufacturing. The saving of material is certainly an advantage of additive manufacturing, as well as the ability to create extremely complex shapes with internal features. Precision is a fundamental element in prosthetics, and from the studies analyzed it is clear that in many situation the precision and accuracy achieved by additive manufacturing are equal or slightly more accurate than subtractive  manufacturing, in a range generally considered suitable for clinical use. \\
Interesting is the possibility of performing an intraoral scan and comparing it digitally with the previous ones; it can be extremely useful in monitoring the growth of the little patient, especially in orthodontics. \\
While the dentist is often already in the conditions to use these technologies, it is important to promote the competence in the management of digital data. Guarantee patient privacy is of primary importance, given how easily digital data can be exchanged. Proper management of medical images and models is important to preserve the original information.\\ Finally, the printing process must be accurate, with great regard to the operating condition of the printer and to the printing parameters. The literature provides general knowledge and useful insights, but these must always be integrated with the manufacturer's instructions and with the specifications of the machine.\\
Future perspectives include the printing of scaffolds for tissue regeneration and, further, the in vitro production of tissues and organs ready for transplantation on the patient. Both of these paths of research use additive manufacturing techniques, and there is already who suggest that in the future tissue printing can be integrated into the clinical routine \parencite{Reference142}. \\
These are certainly stimulating prospective that deserve a more careful discussion, due to the various implications that these technologies may have in the future. The technology progresses quickly together to the related tools and software, so the methods discussed here represent only one of the many approaches that can be used by the clinician. The patient is always at the center of the treatment, therefore every therapeutic choice must be carefully evaluated and put in place only if there are evidences of real improvement in the patient's quality of life.
