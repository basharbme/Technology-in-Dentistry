% Chapter Template

\chapter{Ringraziamenti} % Main chapter title

\label{Ringraziamenti} % Change X to a consecutive number; for referencing this chapter elsewhere, use \ref{ChapterX}
 
 %----------------------------------------------------
 
 Un mio caro amico disse una volta che i sentimenti che si provano verso le persone, se messi nero su bianco non sono comparabili a quelli che si possono raccontare con le parole.\\

Ed io sono d'accordo con lui.\\

Quindi cosa sono queste righe? Ecco, queste righe sono come una foto. Una come le tante che ognuno di noi ha, salvata sul cellulare o stampata ed attaccata su un album fatto a mano con amore.
Queste righe catturano ciò che sento in questo momento per delle persone speciali, che in tanti modi mi sono state vicine in questi anni. Persone con cui ho condiviso momenti che porto nel cuore e che, nonostante tutto, sono ancora qui.
Non mi spiego ancora come facciano a restare, ma sono sicuro che hanno una pazienza incredibile.
Le foto in generale mi piacciono, ma ne ho poche perché raramente mi viene in mente di scattarne.
Quindi vi chiedo di concedermi una foto alternativa.\\

\emph{\textbf{E sorridete, che se no viene male!}}\\


%Con ognuno di voi ho condiviso la pezzi di vita durante questi anni di università. Alcuni vi conoscevo da prima, altri vi ho incontrati durante questo percorso. Ho ricordi di così tanti momenti con voi che si affollano nella mia mente, e in tutti vedo la felicità, l'amicizia e l'amore. Vorrei raccontarvi ogni cosa che ho vissuto con ognuno di voi, ma alla fine mi rendo conto che non ha senso. Non ha senso perché voi già sapete tutto, perché voi eravate li con me in quei momenti.\\

Il primo grazie è per \textbf{i miei genitori}, che continuano sempre a supportarmi nel mio percorso. È grazie a voi che mi ritrovo a scrivere queste righe. Avete sempre messo il nostro bene al centro della famiglia ed è a voi che mi ispiro pensando al futuro. Questo traguardo è vostro più di quanto sia mio, che avete instancabilmente lavorato affinché potesse realizzarsi. Grazie, di tutto!\\

Un ringraziamento va al \textbf{Prof Messina}, che mi ha dato fiducia nella realizzazione di questo progetto di tesi e per i consigli preziosi che mi ha fornito nel corso della sperimentazione e nella stesura. La ringrazio anche per ciò che mi ha trasmesso nel corso di questi anni di università, soprattutto per il metodo di analisi clinica e per avermi mostrato quanto sia importante il rapporto di fiducia tra medico e paziente. \\

\textbf{Pasky}, dovrei ringraziarti per infinite cose, per esserci sempre stato, per le chiacchierate sempre appassionate, per le partite alla Play, i film, la musica e milioni di altre cose; ma alla fine tu lo sai già fratè, queste righe non rendono giustizia. \\

Un ringraziamento è di dovere per \textbf{Rossella}, che negli ultimi anni mi è stata vicina e mi ha incoraggiato ad espandere la mia quotidianità con nuove idee e progetti per il futuro. Il tempo trascorso insieme, seppure non sia mai abbastanza, è sempre ricco di amore, felicità e leggerezza, e per questo è la cosa più preziosa.\\

Grazie anche a miei \textbf{amici e coinquilini} con cui ho condiviso la gran parte della mia permanenza a Palermo. Se tutto è andato bene alla fine, è anche perché tutto è iniziato per il meglio in quella casa in via Marinuzzi 100.\\

Un grazie speciale va a \textbf{Salvo}, che negli ultimi vent'anni c'è sempre stato, per tutto. Se facessi l'elenco di tutte le cose che abbiamo fatto insieme (anche solo di quelle poche che ricordo) questi ringraziamenti sarebbero più lunghi della tesi. Se poi inserissi i progetti abbozzati insieme, bhe a quel punto non basterebbe la carta della copisteria. So che non ci tieni a sti ringraziamenti e magari starai già ridendo, ma io li metto uguale, anche per ricordarmi che dobbiamo aumentare il numero di progetti che portiamo dalla teoria alla pratica! ;)\\

Un altro grazie va alla \emph{\textbf{Disagio House}} ed il popolo che le gravita attorno. Nonostante mi intrufolassi nella casa per diversi giorni, mi avete sempre accolto con un sorriso gentile e con grandi quantità di prelibata pizza bianca e birre artigianali. Io provando a ricambiare vi ho donato delle splendide polpettine alla menta, che poi penso siano state riproposte dalla Colgate per qualche campagna anti-alitosi\ldots in ogni caso vi cedo il Copyright.
Il caso ha voluto che le sessioni di laurea combaciassero nei nostri rispettivi Atenei, per cui non posso che rinnovare i miei migliori auguri a voi nuovi dottori. Ci vediamo presto!\\



\emph{\textbf{Un grande grazie va a tutti i miei colleghi e amici, senza i quali questi anni non sarebbero stati così ricchi di bei momenti.}}\\

\textbf{Tommaso}, grazie per le tue freddure e per i brindisi che abbiamo condiviso. Grazie per le ballate sui tavoli e per avermi raccontato di storia, cultura e religione, facendomi apprezzare ogni angolo di questa città e della nostra tradizione.\\

\textbf{Giovanna}, grazie per la tua (vitale) assistenza organizzativa e per tutte le risate fatte a lezione. Grazie per tutti i brindisi fatti insieme, mentre studiavamo da qualche sbob a me incomprensibile, ma tu avevi sempre tutto pronto, bello ordinato e colorato. Grazie amica, per aver condiviso con me le variegate esperienze di questi anni e per avermi reso una persona migliore.\\

\textbf{Martina}, grazie per le birre e il cibo a qualunque orario, per i video stupidi che abbiamo condiviso e per tutti i passaggi che mi hai dato in questi anni. Grazie per la tua generosità e gentilezza e per il continuo impegno che metti in quello che fai, che siano impegni universitari e professionali o la lotta per la legalità e l'uguaglianza sociale. Non cambiare mai. E grazie per avermi fatto conoscere Andrea, nonostante ciò che continua a dire Giovanna. ;)\\

\textbf{Giuliana}, grazie per il tuo cuore palermitano puro in un vestito da principessa. Grazie per le tue imitazioni e per tutte le volte che millantavi di non sapere nulla dell'esame e puntualmente te ne uscivi con una lode. Grazie per esserci sempre, che sia per un caffè, una pizza (sugo di pomodoro, crudo e scaglie) o un'ansia preesame (che ormai sono finiti! :D ) . E grazie per avermi fatto conoscere una persona speciale come Valerio, sempre pronto per un brindisi e generoso nei consigli.\\

\textbf{Giorgia}, grazie per i negroni, le profonde chiacchierate ed i tuoi giochi improvvisati. Grazie per essere sempre sincera e gentile e per impegnarti sempre ad essere migliore giorno dopo giorno. Migliore lo sei già e rendi migliore chi ti sta intorno.\\

\textbf{Cinzia}, grazie per la tua simpatia e sincerità, per i fotobrodi e le birre alla spina. Grazie per i tuoi consigli e il tuo supporto. Grazie per aver alleviato le lezioni ed i tirocini con ironia e spensieratezza. E grazie anche per avermi fatto conoscere il caro Vicio.\\

\textbf{Peppino e Angelo}, gli ultimi banchi sono una tradizione, così come la fame improvvisa durante le sale operatorie e le pizze mangiate fuori dall'aula. Avete alleggerito quelle aule ripiene di estrogeni ed evidenziatori colorati, e per questo non posso fare a meno di dirvi grazie.\\


A huge thank you to \textbf{my Munich family}, friends who I never hoped to find and that instead made me go through an unforgettable year. This project is also yours! Thanks for the time that we spent together, for the parties, for the BBQ in the garden and at the lake, for the lectures that we attended and for the always interesting discussions that we did about any topic! Thanks for the nights at the Blitz and for every time that you entered in my room with someting to eat. We passed good and bad times, but we were always together, like a Family. Thanks Bros, love you all!\\

\pagebreak

Un ultimo ringraziamento lo devo fare non ad una persona in particolare, ma a tutte le persone che ho incontrato nella mia vita, docenti, parenti, professionisti, che mi hanno detto che c'era qualcosa che non avrei potuto fare, perché non ne sarei stato in grado, perché avrei fatto brutta figura, perché non è quello che ci si aspetta\ldots Questo sarà anche vero nella maggior parte dei casi, ma provare è una mia scelta, una scelta che ognuno ha, perché ognuno è libero di provare e fallire, per poi provare ancora.\\
Sono felice di non aver dato ascolto a queste voci, sono felice di aver sbagliato molte più volte di quelle che mi avevano detto che avrei sbagliato. Perché così facendo ho imparato a far tesoro di ogni sbaglio.\\ Grazie a chi non ha creduto in me, siete stati la molla che mi ha fatto andare avanti ogni volta che non avrei più voluto farlo.\\ Non date retta a chi dice che non siete in grado di fare qualcosa che voi ardentemente volete, ma impiegate le vostre energie per raggiungere l'obiettivo.\\

\emph{\textbf{La nostra vita è solo nostra, non lasciamo che altri ne decidano il percorso}}.


